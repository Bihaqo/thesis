\documentclass{article} % For LaTeX2e
\usepackage{hyperref}
\usepackage{url}
\usepackage[russian]{babel}
\usepackage[utf8]{inputenc}


\title{Тензорные методы в машинном обучении}


\author{Александр Новиков}

\begin{document}

\maketitle

\section{Введение}
\section{Тензорные методы в машинном обучении}
\section{Разложение в Тензорный Поезд}
\section{Вывод МРФ}
\subsection{МРФ}
\subsection{Задача вывода}
\subsection{Обзор основных подходов}
\subsection{Тензорный подход}
\section{Сжатие нейросетевых моделей}
\subsection{Нейросети}
\subsection{Основные методы сжатия}
\subsection{Тензорный подход к сжатию}
\section{Ядерный метод на основе ТТ}
\subsection{Полиномиальные методы в машинном обучении}
\subsection{Основные подходы к обучению полиномиальных моделей}
\subsection{Тензорный подход к полиномиальным моделям}
\subsection{Риманова оптимизация}
\section{Комплексы программ}
\subsection{Библиотека тензорных вычислений t3f}
\subsection{Реализация алгоритма вывода в МРФ}
\subsection{Реализация алгоритма сжатия нейросетей}
\subsection{Реализация тензорных полиномиальных моделей}
\section{Результаты численных экспериментов}
\section{Заключение}
\section{Список литературы}
\section{Публикации автора по теме диссертации}

\end{document}
