% \cite[с.~54]{Sokolov}
\chapter{Основные понятия} \label{chap:definitions}

\section{Разложение в тензорный поезд} \label{sec:tt-decomposition}

\begin{figure}[ht]
  \begin{center}
    \def\svgwidth{10cm}
    \normalsize
    \includesvg{images/TT}
    \caption{Иллюстрация разложения в Тензорный Поезд для тензора $\tens{A}$ размера $3 \times 4 \times 4 \times 3$ все ТТ-ранги которого равны $3$. \label{fig:TT}}
    \end{center}
\end{figure}

Будем говорить, что $n$-мерный тензор~$\mathbf{A}$ представлен в TT\hyp{}формате, если для всех размерностей~$i=1,\ldots,n$ и всех значений индексов по этой размерности~$x_i = 1, \ldots, d_i$ ($d = \max_{i=1,\ldots,n} d_i$) существуют матрицы~$G_i^{\mathbf{A}}[x_i]$, такие, что каждый элемент тензора~$\mathbf{A}$ представим в виде произведения матриц (см Рис.~\ref{fig:TT}):
\begin{equation}
\label{TT-format}
\mathbf{A}(x_1, \dots, x_n) = G_1^{\mathbf{A}}[x_1] G^{\mathbf{A}}_2[x_2] \dotsm G^{\mathbf{A}}_n[x_n].
\end{equation}
При этом все матрицы относящиеся к одной и той же размерности $i$ должны иметь одинаковые размеры~$\rank_{i-1}(\mathbf{A}) \times \rank_i(\mathbf{A})$. Чтобы результат матричного произведения~\eqref{TT-format} был равен числу, положим~$\rank_{0}(\mathbf{A}) = \rank_{n}(\mathbf{A}) = 1$. Последовательность~$\left\{\rank_i(\mathbf{A})\right\}_{i=0}^n$ будем называть \emph{TT\hyp{}рангами}~тензора $\mathbf{A}$, а максимальный элемент последовательности~--- \emph{максимальным TT\hyp{}рангом} тензора~$\mathbf{A}$:~$\rank(\mathbf{A}) = \max_{i = 0, \dots, n} \rank_i(\mathbf{A})$.
Набор матриц~$G^{\mathbf{A}}_i$, соответствующих одному измерению, называется \emph{TT\hyp{}ядром} тензора~$\mathbf{A}$.
Представление тензора в TT\hyp{}формате будем называть \emph{TT\hyp{}разложением} или \emph{TT\hyp{}представлением}.

Для любого $n$-мерного тензора~$\mathbf{A}$ существует TT\hyp{}представление с максимальным TT\hyp{}рангом $\rank(\mathbf{A}) \leq d^{\frac{n}{2}}$~(см. теорему~2.1 Оселедца \cite{oseledets2011ttMain}). Отметим, что TT\hyp{}представление тензора не единственно.


Для обозначения~$(\alpha_{i-1}, \alpha_{i})$-ого элемента матрицы~$G_i^{\mathbf{A}}[x_i]$ будет использоваться символ~$G_i^{\mathbf{A}}[x_i](\alpha_{i-1}, \alpha_{i})$.
Пользуясь определением произведения матриц, можно переписать формулу~\eqref{TT-format} через элементы TT\hyp{}ядер:

\begin{equation}
	\label{TT-format-sum}
	\mathbf{A}(\vec{x}) =
	\sum_{\alpha_0, \ldots, \alpha_n} G^{\mathbf{A}}_1[x_1](\alpha_0, \alpha_1) \ldots G^{\mathbf{A}}_n[x_n](\alpha_{n-1}, \alpha_n).
\end{equation}

Для хранения всех элементов тензора~$\mathbf{A}$ требуется~$\prod_{i=1}^n d_i$ ячеек памяти, тогда как хранение~$\mathbf{A}$ в TT\hyp{}формате требует~$\sum_{i=1}^n d_i \rank_{i-1}(\mathbf{A}) \rank_{i}(\mathbf{A})$. Таким образом, TT\hyp{}представление тензора с низкими TT\hyp{}рангами существенно компактнее перечисления его элементов.

Существуют две различные постановки задачи перевода тензора в TT\hyp{}формат: точное TT\hyp{}представление (алгоритм TT-SVD~\cite{oseledets2011ttMain}, применимый для небольших тензоров), и построение приближенного TT\hyp{}представления по небольшому подмножеству элементов тензора. Наилучшим из алгоритмов второго класса в настоящее время является метод AMEn-cross~\cite{dolgov2013amenCross}.

\begin{table}[t]
\caption{Операции, которые можно эффективно выполнять над тензорами в TT\hyp{}формате. Для каждой операции указана её вычислительная сложность и TT\hyp{}ранг результата для ситуаций, когда результат является тензором в TT\hyp{}формате.}
\label{TT-tensor-operations}
\vskip 0.05in
\begin{center}
\begin{small}
% \begin{sc}
\begin{tabular}{l@{\;\;}l@{\;\;}l}
\hline
Операция & Ранг результата & Вычислительная сложность \\
\hline
% \abovespace
$\mathbf{C} = \mathbf{A} \cdot \const$      & $\rank(\mathbf{C}) = \rank(\mathbf{A})$             & $O(d  \rank(\mathbf{A}))$\\
$\mathbf{C} = \mathbf{A} + \const$          & $\rank(\mathbf{C}) = \rank(\mathbf{A})\!+\!1$         & $O(n  d  \rank^2(\mathbf{A}))$\\
$\mathbf{C} = \mathbf{A} + \mathbf{B}$           & $\rank(\mathbf{C}) \leq \rank(\mathbf{A})\!+\!\rank(\mathbf{B})$       & $O(n  d \, (\rank(\mathbf{A}) + \rank(\mathbf{B}))^2)$\\
$\mathbf{C} = \mathbf{A} \odot \mathbf{B}$       & $\rank(\mathbf{C}) \leq \rank(\mathbf{A})  \rank(\mathbf{B})$   & $O(n  d  \rank^2(\mathbf{A})  \rank^2(\mathbf{B}))$\\
$\vec{c}=M\vec{b}$                          & $\rank(\vec{c}) \leq \rank(M) \rank(\vec{b})$       & $O(n d^2 \rank^2(M) \rank^2(\vec{b}))$\\
%\belowspace
$\sumall \mathbf{A}$                & --                      & $O(n  d  \rank^2(\mathbf{A}))$\\
$\left \| \mathbf{A} \right \|_F$       & --                      & $O(n  d  \rank^3(\mathbf{A}))$\\
% \belowspace
$\mathbf{C} = \round(\mathbf{A}, \varepsilon)$ & $\rank(\mathbf{C}) \leq \rank(\mathbf{A})$          & $O(n d \rank^3(\mathbf{A}))$\\
\hline
\end{tabular}
% \end{sc}
\end{small}
\end{center}
\vskip -0.1in
\end{table}

Одним из основных достоинств TT\hyp{}формата является возможность эффективно применять различные операции к тензорам в TT\hyp{}формате:
умножение тензора на константу, добавление константы к тензору, поточечное сложение и умножение тензоров (результат этих операций~--- это тензор в TT\hyp{}формате с возросшими TT\hyp{}рангами); подсчет глобальных характеристик тензора, таких как сумма всех элементов или норма Фробениуса. Обзор операций, которые используются в этой статье, приведен в таблице~\ref{TT-tensor-operations} (детальный обзор проведен Оселедцом~\cite{oseledets2011ttMain}).
Примение операций к TT\hyp{}тензорам увеличивает TT\hyp{}ранги даже в том случае, когда существует низкоранговое TT\hyp{}представление результата. Чтобы контролировать рост TT\hyp{}рангов существует операция TT\hyp{}округления.
По TT\hyp{}представлению тензора~$\mathbf{A}$ и относительной точности~$\varepsilon \geq 0$ операция TT\hyp{}округления $\round(\mathbf{A},\varepsilon)$ найдет тензор~$\widehat{\mathbf{A}}$ в TT\hyp{}формате который, во-первых, достаточно близок к тензору~$\mathbf{A}$: $\| \mathbf{A} - \mathbf{\widehat{A}} \|_F \leq \varepsilon  \| \mathbf{A}  \|_F$ и, во-вторых, обладает минимальными TT\hyp{}рангами среди всех тензоров~$\mathbf{B}: \| \mathbf{A} - \mathbf{B} \|_F \leq \frac{\varepsilon}{\sqrt{n - 1}}  \| \mathbf{A}  \|_F$. Наличие операции TT\hyp{}округления позволяет применять последовательность операций к тензорам (например, округляя результат после применения каждой операции), контролируя рост TT\hyp{}рангов.


Для повышения эффективности работы с векторами и матрицами специальным образом вводятся понятия TT\hyp{}формата вектора и ТТ\hyp{}формата матрицы.
Пусть существует отображение между индексами вектора~$\vec{b}\in\mathbb{R}^q$ и $n$-мерными векторами $\vec{y} = (y_1, \dots, y_n)$\footnote{
Количество элементов в векторе~$\vec{b}$ равно $\prod_{i = 1}^n d_i$.
}.
\emph{TT\hyp{}представлением вектора~$\vec{b}$} называется TT\hyp{}представление тензора~$\mathbf{B}(y_1, \dots, y_n)$, содержащего все элементы~$\vec{b}$.


Определим понятие TT\hyp{}формата для матриц. Пусть существует отображение между индексами строк и столбцов матрицы~$M$ в $n$-мерные вектора $\vec{x}$ и $\vec{y}$ соответственно. Переупорядочим размерности и представим получившийся тензор в TT\hyp{}формате:
\begin{equation*}
\mathbf{M}((x_1, y_1),\ldots,(x_n,y_n))=G_1^{\mathbf{M}}[x_1,y_1]\dots G_n^{\mathbf{M}}[x_n,y_n],
\end{equation*}
где $G_i$, $i = 1,\dots,n$ --- это TT\hyp{}ядра, а~$G_i^{\mathbf{M}}[x_i,y_i]$ --- матрицы. \emph{TT\hyp{}представлением матрицы~$M$} будем называть TT\hyp{}представление тензора~$\mathbf{M}$. Отметим, что матрица в TT\hyp{}формате не обязана быть квадратной, т.~к. $x_i$ и $y_i$ могут принимать разное число возможных значений


Для матрицы~$M$ и вектора~$\vec{b}$ представленных в TT\hyp{}формате можно эффективно вычислять произведение $\vec{c}=M\vec{b}$ (если соответствующие размерности совпадают). Результатом этой операции является вектор~$\vec{c}$ в TT\hyp{}формате с рангами равными произведению рангов~$M$ и $\vec{b}$: $\rank_i(\vec{c}) = \rank_i(M) \rank_i(\vec{b})$.

Наличие специального определения TT\hyp{}формата для векторов и матриц позволяет применять операции линейной алгебры к задачам большого размера. Например, для поиска минимального элемента тензора можно вытянуть его в диагональную матрицу и применить приближенный метод поиска минимальных собственных значений основанный на алгоритме DMRG~\cite{khoromskij2010dmrg}.

\section{Машинное обучение} \label{sec:ml}
