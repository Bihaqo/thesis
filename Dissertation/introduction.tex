\chapter*{Введение}							% Заголовок
\addcontentsline{toc}{chapter}{Введение}	% Добавляем его в оглавление

\section{Актуальность темы исследования} \label{sec:intro-importance}
Машинное обучение переживает период бурного развития и за последние годы было решено много важных на практике проблем, таких как распознавание объектов изображенных на фотографиях, сегментация фотографий, машинный перевод, распознфвание речи, и др. Тем не менее, алгоритмы показывающие наилучшее качество во многих задачах (такие как нейронные сети и Марковские Случайные Поля) вычислительно затратны как во время обучения, так и во время использования моделей. Так, для обучения модели машинного перевода может требоваться месяцы времени на вычислительном кластере~\cite{}, а итоговая модель может занимать сотни мегабайт и требовать \alert{ФЛОПС?} для обработки одного предложения, что оказывается слишком затратно для использования на мобильных и встроенных устройствах, как по части быстродействия и требования к памяти, так и по части использования ресурса батарейки. В связи с этим возникает потребность к разработке новых более быстрых методов обучения моделей, способов сжатия и ускорения моделей, и разработки новых менее ресурсо-затратных моделей с нуля. По всем этим направлениям ведуться активные исследования в лучших лабораториях машинного обучения в мире.

В данной диссертации исследуется возможность применения алгоритмов малорангового сжатия тензоров для решения озвученных выше задач. 
\section{Историческая справка} \label{sec:intro-history}
Про развитие МЛ, про тензорные методы в МЛ
\section{Общая характеристика кандидатской работы} \label{sec:intro-overview}
\subsection{Научная новизна} \label{sec:intro-novelity}

\subsection{Практическая ценность} \label{sec:intro-appliability}
\subsection{Положения, выносимые на защиту} \label{sec:intro-defended-topics}
\begin{enumerate}
	\item Комплекс программ для работы с разложением в тензорный поезд на языке Python с поддержкой графических ускорителей;
	\item Вычислительный метод оценки нормировочной константы марковского случайного поля;
	\item Вычислительный метод оценки поиска собственного разложения матриц в ТТ-формате в приложении к поиску конфигурации минимума энергии Марковского случайного поля;
	\item Модель Искусственной Нейронной сети опережающая аналоги по соотношению число настраиваемых параметров к качеству работы.
	\item Модель Рекуррентной Искусственной Нейронной Сети для которой возможно быстрое обучение с помощью Римановой оптимизации.
\end{enumerate}
\subsection{Апробация работы и публикации} \label{sec:intro-publications}
\subsection{Личный вклад} \label{sec:intro-publications}
\section{Содержание работы по главам} \label{sec:intro-detailed-table-of-contents}
\section{Благодарности} \label{sec:intro-acknowledgment}

\newcommand{\actuality}{}
\newcommand{\progress}{}
\newcommand{\aim}{{\textbf\aimTXT}}
\newcommand{\tasks}{\textbf{\tasksTXT}}
\newcommand{\novelty}{\textbf{\noveltyTXT}}
\newcommand{\influence}{\textbf{\influenceTXT}}
\newcommand{\methods}{\textbf{\methodsTXT}}
\newcommand{\defpositions}{\textbf{\defpositionsTXT}}
\newcommand{\reliability}{\textbf{\reliabilityTXT}}
\newcommand{\probation}{\textbf{\probationTXT}}
\newcommand{\contribution}{\textbf{\contributionTXT}}
\newcommand{\publications}{\textbf{\publicationsTXT}}


{\actuality} Обзор, введение в тему, обозначение места данной работы в
мировых исследованиях и~т.\:п., можно использовать ссылки на другие
работы~\cite{Gosele1999161} (если их нет, то в автореферате
автоматически пропадёт раздел <<Список литературы>>). Внимание! Ссылки
на другие работы в разделе общей характеристики работы можно
использовать только при использовании \verb!biblatex! (из-за технических
ограничений \verb!bibtex8!. Это связано с тем, что одна и та же
характеристика используются и в тексте диссертации, и в
автореферате. В последнем, согласно ГОСТ, должен присутствовать список
работ автора по теме диссертации, а \verb!bibtex8! не умеет выводить в одном
файле два списка литературы).

Для генерации содержимого титульного листа автореферата, диссертации и
презентации используются данные из файла \verb!common/data.tex!. Если,
например, вы меняете название диссертации, то оно автоматически
появится в итоговых файлах после очередного запуска \LaTeX. Согласно
ГОСТ 7.0.11-2011 <<5.1.1 Титульный лист является первой страницей
диссертации, служит источником информации, необходимой для обработки и
поиска документа.>> Наличие логотипа организации на титульном листе
упрощает обработку и поиск, для этого разметите логотип вашей
организации в папке images в формате PDF (лучше найти его в векторном
варианте, чтобы он хорошо смотрелся при печати) под именем
\verb!logo.pdf!. Настроить размер изображения с логотипом можно в
соответствующих местах файлов \verb!title.tex!  отдельно для
диссертации и автореферата. Если вам логотип не нужен, то просто
удалите файл с логотипом.

% {\progress} 
% Этот раздел должен быть отдельным структурным элементом по
% ГОСТ, но он, как правило, включается в описание актуальности
% темы. Нужен он отдельным структурынм элемементом или нет ---
% смотрите другие диссертации вашего совета, скорее всего не нужен.

{\aim} данной работы является \ldots

Для~достижения поставленной цели необходимо было решить следующие {\tasks}:
\begin{enumerate}
  \item Исследовать, разработать, вычислить и~т.\:д. и~т.\:п.
  \item Исследовать, разработать, вычислить и~т.\:д. и~т.\:п.
  \item Исследовать, разработать, вычислить и~т.\:д. и~т.\:п.
  \item Исследовать, разработать, вычислить и~т.\:д. и~т.\:п.
\end{enumerate}


{\novelty}
\begin{enumerate}
  \item Впервые \ldots
  \item Впервые \ldots
  \item Было выполнено оригинальное исследование \ldots
\end{enumerate}

{\influence} \ldots

{\methods} \ldots

{\defpositions}
\begin{enumerate}
  \item Первое положение
  \item Второе положение
  \item Третье положение
  \item Четвертое положение
\end{enumerate}
В папке Documents можно ознакомиться в решением совета из Томского ГУ
в файле \verb+Def_positions.pdf+, где обоснованно даются рекомендации
по формулировкам защищаемых положений. 

{\reliability} полученных результатов обеспечивается \ldots \ Результаты находятся в соответствии с результатами, полученными другими авторами.


{\probation}
Основные результаты работы докладывались~на:
перечисление основных конференций, симпозиумов и~т.\:п.

{\contribution} Автор принимал активное участие \ldots

%\publications\ Основные результаты по теме диссертации изложены в ХХ печатных изданиях~\cite{Sokolov,Gaidaenko,Lermontov,Management},
%Х из которых изданы в журналах, рекомендованных ВАК~\cite{Sokolov,Gaidaenko}, 
%ХХ --- в тезисах докладов~\cite{Lermontov,Management}.

\ifnumequal{\value{bibliosel}}{0}{% Встроенная реализация с загрузкой файла через движок bibtex8
    \publications\ Основные результаты по теме диссертации изложены в XX печатных изданиях, 
    X из которых изданы в журналах, рекомендованных ВАК, 
    X "--- в тезисах докладов.%
}{% Реализация пакетом biblatex через движок biber
%Сделана отдельная секция, чтобы не отображались в списке цитированных материалов
    \begin{refsection}[vak,papers,conf]% Подсчет и нумерация авторских работ. Засчитываются только те, которые были прописаны внутри \nocite{}.
        %Чтобы сменить порядок разделов в сгрупированном списке литературы необходимо перетасовать следующие три строчки, а также команды в разделе \newcommand*{\insertbiblioauthorgrouped} в файле biblio/biblatex.tex
        \printbibliography[heading=countauthorvak, env=countauthorvak, keyword=biblioauthorvak, section=1]%
        \printbibliography[heading=countauthorconf, env=countauthorconf, keyword=biblioauthorconf, section=1]%
        \printbibliography[heading=countauthornotvak, env=countauthornotvak, keyword=biblioauthornotvak, section=1]%
        \printbibliography[heading=countauthor, env=countauthor, keyword=biblioauthor, section=1]%
        \nocite{%Порядок перечисления в этом блоке определяет порядок вывода в списке публикаций автора
                vakbib1,vakbib2,%
                confbib1,confbib2,%
                bib1,bib2,%
        }%
        \publications\ Основные результаты по теме диссертации изложены в \arabic{citeauthor} печатных изданиях, 
        \arabic{citeauthorvak} из которых изданы в журналах, рекомендованных ВАК, 
        \arabic{citeauthorconf} "--- в тезисах докладов.
    \end{refsection}
    \begin{refsection}[vak,papers,conf]%Блок, позволяющий отобрать из всех работ автора наиболее значимые, и только их вывести в автореферате, но считать в блоке выше общее число работ
        \printbibliography[heading=countauthorvak, env=countauthorvak, keyword=biblioauthorvak, section=2]%
        \printbibliography[heading=countauthornotvak, env=countauthornotvak, keyword=biblioauthornotvak, section=2]%
        \printbibliography[heading=countauthorconf, env=countauthorconf, keyword=biblioauthorconf, section=2]%
        \printbibliography[heading=countauthor, env=countauthor, keyword=biblioauthor, section=2]%
        \nocite{vakbib2}%vak
        \nocite{bib1}%notvak
        \nocite{confbib1}%conf
    \end{refsection}
}
При использовании пакета \verb!biblatex! для автоматического подсчёта
количества публикаций автора по теме диссертации, необходимо
их здесь перечислить с использованием команды \verb!\nocite!.
    

 % Характеристика работы по структуре во введении и в автореферате не отличается (ГОСТ Р 7.0.11, пункты 5.3.1 и 9.2.1), потому её загружаем из одного и того же внешнего файла, предварительно задав форму выделения некоторым параметрам

% \textbf{Объем и структура работы.} Диссертация состоит из~введения, четырёх глав, заключения и~двух приложений.
%% на случай ошибок оставляю исходный кусок на месте, закомментированным
%Полный объём диссертации составляет  \ref*{TotPages}~страницу с~\totalfigures{}~рисунками и~\totaltables{}~таблицами. Список литературы содержит \total{citenum}~наименований.
%
%
%
%Matrix and tensor decompositions were recently used to speed up the inference time of convolutional neural networks~\cite{Denil2014speedup,lebedev2014speeding}. While we focus on fully-connected layers, \cite{lebedev2014speeding} used the CP-decomposition to compress a $4$-dimensional convolution kernel and then used the properties of the decomposition to speed up the inference time. Their work share the same spirit with our method and the approaches can be readily combined.

Полный объём диссертации составляет
\formbytotal{TotPages}{страниц}{у}{ы}{}, включая
\formbytotal{totalcount@figure}{рисун}{ок}{ка}{ков} и
\formbytotal{totalcount@table}{таблиц}{у}{ы}{}.   Список литературы содержит  
\formbytotal{citenum}{наименован}{ие}{ия}{ий}.
